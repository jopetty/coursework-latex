\documentclass[cmu]{pset}

\title{\#1}
\name{Jackson Petty}
\course{Abstract Algebra}
\due{\today}

\begin{document}

\begin{problem}
Prove that there are infinitely many prime numbers.
\end{problem}
\begin{proof}
Let $S$ be a set of finitely many primes. Let $q$ be one more than the product of all elements of $S$. Consider that if $q$ is prime, then there must obviously be at least one prime number which is not enumerated in $S$. On the other hand, if $q$ is not prime, then by the fundamental theorem of algebra there must exist at least one prime factor of $q$, and this factor is not included in $S$ since $q$ is not divisible by any elements of $S$. Thus any finite enumeration of primes is necessarily incomplete.
\end{proof}

\begin{problem}
What is the air-speed velocity of an unladen swallow?
\end{problem}
\begin{solution}
African or European?
\end{solution}

\begin{exercise}
Go outside and find some fossils, because fossils are cool.
\end{exercise}

\begin{solution}
\[ \int_0^{\pi/6} \sec y \dd{y} = \ln\sqrt{3} \cdot i^{64} \qedhere \]
\end{solution}

As any dedicated reader can clearly see, the Ideal of practical reason is a representation of, as far as I know, the things in themselves; as I have shown elsewhere, the phenomena should only be used as a canon for our understanding. The paralogisms of practical reason are what first give rise to the architectonic of practical reason. As will easily be shown in the next section, reason would thereby be made to contradict, in view of these considerations, the Ideal of practical reason, yet the manifold depends on the phenomena. Necessity depends on, when thus treated as the practical employment of the never-ending regress in the series of empirical conditions, time. Human reason depends on our sense perceptions, by means of analytic unity. There can be no doubt that the objects in space and time are what first give rise to human.

\begin{problem}
Suppose that $H$ is a proper subgroup of $G$. Prove that 
\[ G \not= \bigcup_{x \in G} xHx^{-1}. \]
\end{problem}

As any dedicated reader can clearly see, the Ideal of practical reason is a representation of, as far as I know, the things in themselves; as I have shown elsewhere, the phenomena should only be used as a canon for our understanding. The paralogisms of practical reason are what first give rise to the architectonic of practical reason. As will easily be shown in the next section, reason would thereby be made to contradict, in view of these considerations, the Ideal of practical reason, yet the manifold depends on the phenomena. Necessity depends on, when thus treated as the practical employment of the never-ending regress in the series of empirical conditions, time. Human reason depends on our sense perceptions, by means of analytic unity. There can be no doubt that the objects in space and time are what first give rise to human.

\[ \mathsf{S}_{\mu\nu}{}^{\lambda} \qquad \tensor{S}{_{\mu\nu}^{\lambda}} \]

As any dedicated reader can clearly see, the Ideal of practical reason is a representation of, as far as I know, the things in themselves; as I have shown elsewhere, the phenomena should only be used as a canon for our understanding. The paralogisms of practical reason are what first give rise to the architectonic of practical reason. As will easily be shown in the next section, reason would thereby be made to contradict, in view of these considerations, the Ideal of practical reason, yet the manifold depends on the phenomena. Necessity depends on, when thus treated as the practical employment of the never-ending regress in the series of empirical conditions, time. Human reason depends on our sense perceptions, by means of analytic unity. There can be no doubt that the objects in space and time are what first give rise to human.

\begin{theorem}[Asaki]
As any dedicated reader can clearly see, the Ideal of practical reason is a representation of, as far as I know, the things in themselves; as I have shown elsewhere, the phenomena should only be used as a canon for our understanding. The paralogisms of practical reason are what first give rise to the architectonic of practical reason. As will easily be shown in the next section, reason would thereby be made to contradict, in view of these considerations, the Ideal of practical reason, yet the manifold depends on the phenomena. Necessity depends on, when thus treated as the practical employment of the never-ending regress in the series of empirical conditions, time. Human reason depends on our sense perceptions, by means of analytic unity. There can be no doubt that the objects in space and time are what first give rise to human.
\end{theorem}

\begin{problems}
	\problem Suppose that $H$ is a proper subgroup of $G$. Prove that 
	\[ G \not= \bigcup_{x \in G} xHx^{-1}. \]
	As any dedicated reader can clearly see, the Ideal of practical reason is a representation of, as far as I know, the things in themselves;

	\begin{problem}
	Throughout, let $G$ be a group and let $N \mathrel{\unlhd} G$ be a normal subgroup.
	\begin{parts}
		\part This is a part
		\part this is another part
	\end{parts}
	\end{problem}

	\item As any dedicated reader can clearly see, the Ideal of practical reason is a representation of, as far as I know, the things in themselves;
\end{problems}

\begin{problem}
	Suppose that $H$ is a proper subgroup of $G$. Prove that 
\[ G \not= \bigcup_{x \in G} xHx^{-1}. \]
\end{problem}

\end{document}